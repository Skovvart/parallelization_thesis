% !TeX root = ../thesis.tex
\subsection{CUDA C Solution}
A CUDA C solution was implemented to be used as a base-line comparison for the other CUDA-based solutions.
It is intended to make full use of parallelization and it should be fairly efficient but little time has been spent optimizing it.
It is based on the initial C\# solution but features differences due to language-restrictions and the paradigm-switch to parallel computing.
It features the same 6 life insurance plans as the C\# version, but a key difference is that every plan has its own implementation of a Runge-Kutta 4 kernel which is identical but for references to implementations of different $dV$ and $bj\_ii$ functions and the number of states used by the plan.
This is due to the fact that CUDA C kernels do not allow references to objects or methods as parameters. %unless certain version of cuda?
Another difference is that all of the kernel code is required to be in the same file as no cross-referencing is possible unlike the C\# version.

See code sample \ref{cuda_pureendowment} for an implementation of the Pure Endowment insurance plan in CUDA C.
\begin{lstlisting}[language=cudac, caption=The pure endowment insurance plan expressed in CUDA C, label=cuda_pureendowment]
__device__ class PureEndowment {
	__device__ floatP b_0(floatP t) { return (floatP)0; }
	__device__ floatP mu_01(floatP t) { return GM(t); }
	__device__ floatP bj_00(floatP t) { return t == pensiontime ? bpension : (floatP)0; }
	__device__ floatP bj_01(floatP t) { return (floatP)0; }
public:
	#define PESTATES 1
	__device__ void dV(floatP t, floatP *V, floatP *res){ 
		res[0] = r(t) * V[0] - b_0(t) - mu_01(t) * (0 - V[0] + bj_01(t));
	}

	__device__ void bj_ii(floatP t, floatP *res){
		res[0] = bj_00(t);
	}
} pe;
\end{lstlisting}

Note that this code uses the $floatP$ type alias to easily switch between single and double precision.
What may be of interest is the that the $pe$ object is copied to device memory and that a compile-time replaced $PESTATES$ macro is defined.
The device-size object is then referenced in the plan-specific implementation of the Runge-Kutta 4 solver.
The macro is due to C/C++ requiring constant values for array initialization.
Dynamic memory allocation and deallocation of the temporary arrays could be used, but testing showed that it had a significant negative implication on the runtime, especially at higher thread and block configurations.

The single version of the kernels all used less than 32 registers and could thus use all 1024 threads per block, but the double version used too many registers and could only use 512 threads. 
It is possible to limit the amount of registers during compilation and make the remaining registers spill to local (global) memory, but this is often not preferable as global memory is a lot slower than registers.

Running one iteration with single-precision floats took \emph{42.0} ms making it about 65\% slower than the C\# equivalent.
The double-precision took \emph{124.48} ms making it about five times slower than the C\# equivalent.
The key factor to consider is the fact that the parallelized version was never intended to be faster with a single iteration.
Where the single-threaded version scales linearly with the amount of iterations, the parallelized version scales greatly until the resources of all the multiprocessors are used and then linearly at a much higher rate than the single-threaded version afterwards.

The highest performance obtained with single precision was \emph{91.43} iterations per millisecond, making it more than 2200 times faster than the C\# single precision version.
The highest performance with double precision was \emph{25.97} iterations per millisecond, making it more than 650 times faster than the C\# double precision version.
%The double precision version, while respectable, was a good bit slower than could be expected and could most likely be optimized to a rate where it would be about half as fast as the float precision equivalent.

For the full single-precision results see table \ref{table:cudacfloattime} and for double-precision results see table \ref{table:cudacdoubletime}.
Looking at the full CUDA C performance comparisons available in appendix \ref{app:cuda_runtimes}, it is quite clear that there is little point in using configuration with less than 14 blocks (the number of SMs) and 32 threads (the size of a warp).
What may also be of special interest is the diagonals where the total thread amount ($blocks \cdot threads$) is the same.
Here a trend can be spotted where doubling the number of threads appears to be preferable to doubling the number of blocks, though not by a large factor.

For the full CUDA C performance comparisons, see appendix \ref{app:cuda_runtimes}.
For talk of the validity of the results compared to the original C\# solution, see section \ref{subsec:result_comparison}.

\begin{table}[h!]
\centering
{\setlength{\extrarowheight}{2pt}{\setlength{\tabcolsep}{3pt}
\begin{tabular}{ | r | r | r | r | r | r | r | r | }
  \hline
\diaghead{Threads/Blocks}{Threads}{Blocks}
		&	1		&	14		&	14*5	&	14*10	&	14*20	&	14*25	&	14*30	\\ \hline
 1		&	0.02	&	0.33 	&	1.55	&	1.56 	&	2.00 	&	1.92 	&	2.17	\\ \hline
 8		&	0.19	&	2.65 	&	12.39	&	12.45	&	15.94	&	15.35	&	17.35	\\ \hline
 16		&	0.38	&	5.28 	&	24.69	&	24.81	&	31.58	&	30.59	&	34.61	\\ \hline
 32		&	0.75	&	10.48	&	49.02	&	49.26	&	63.16	&	60.76	&	68.91	\\ \hline
 64		&	1.50	&	20.58	&	64.17	&	70.13	&	81.71	&	80.77	&	86.85	\\ \hline
 128 	&	2.95	&	35.83	&	72.14	&	80.42	&	83.14	&	85.78	&	85.69	\\ \hline
 256	&	5.24 	&	44.15	&	78.57	&	90.82	&	91.07	&	88.55	&	89.01	\\ \hline
 512	&	6.40	&	45.49	&	91.03	&	91.35	&	91.41	&	91.40	&	91.43	\\ \hline
1024	&	6.47	&	90.15	&	90.52	&	90.58	&	90.61	&	90.62	&	90.63	\\ \hline
\end{tabular}}}
\caption{CUDA C calculations per ms with single precision\label{table:cudacfloattime}}
\end{table}


\begin{table}[h!]
\centering
{\setlength{\extrarowheight}{2pt}{\setlength{\tabcolsep}{3pt}
\begin{tabular}{ | r | r | r | r | r | r | r | r | }
  \hline
\diaghead{Threads/Blocks}{Threads}{Blocks}
						&	 1		&	  14		&	14*5 	&	14*10	&	14*20	&	14*25	&	14*30	 \\ \hline
1						&	0.01	&	0.11 	&	0.51 	&	0.50 	&	0.65 	&	0.61 	&	0.69 	 \\ \hline
8						&	0.06	&	0.90 	&	4.06 	&	4.01 	&	5.16 	&	4.89 	&	5.50 	 \\ \hline
16						&	0.13	&	1.79 	&	8.11 	&	8.02 	&	10.33	&	9.77 	&	11.01 	 \\ \hline
32						&	0.26	&	3.59 	&	16.22	&	16.03	&	20.65	&	19.53	&	22.01	 \\ \hline
64						&	0.51	&	6.94 	&	18.49	&	21.71	&	23.59	&	24.10	&	25.12	 \\ \hline
128						&	1.00	&	11.52	&	22.74	&	25.00	&	25.30	&	25.25	&	25.69	 \\ \hline
256						&	1.67	&	12.98	&	25.55	&	25.59	&	25.93	&	25.88	&	25.97	 \\ \hline
512						&	1.85	&	25.95	&	25.96	&	25.97	&	25.97	&	25.97	&	25.97	 \\ \hline
1024 (forced registers)	&	1.79	&	24.98	&	25.00	&	25.00	&	25.00	&	25.00	&	25.00	 \\ \hline
\end{tabular}}}
\caption{CUDA C calculations per ms with double precision\label{table:cudacdoubletime}}
\end{table}