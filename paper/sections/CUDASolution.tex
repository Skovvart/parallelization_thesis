% !TeX root = ../thesis.tex
\subsection{CUDA C Solution}
A CUDA C solution was implemented to be used as a base-line comparison for further development.
It is intended to be fairly efficient but not much time has been spent optimizing it, as this has never been the end-goal.
It is based on the initial C\# solution but has some alterations due to the language.
It features the same 6 life insurance plans as the C\# version, but a key difference is that every plan has its own implementation of a Runge-Kutta 4 kernel which is identical but for references to implementations of different $dV$ and $bj\_ii$ functions and the number of states used by the plan.
This is due to the CUDA C kernels not allowing references to objects or methods. %unless certain version of cuda?
Another difference is that all of the kernel code is required to be in the same file as no cross-referencing is possible unlike the C\# version.

See code sample \ref{cuda_pureendowment} for an implementation of Pure Endowment in CUDA C.
\begin{lstlisting}[language=cudac, caption=The pure endowment insurance plan expressed in CUDA C, label=cuda_pureendowment]
__device__ class PureEndowment {
	__device__ floatP b_0(floatP t) { return (floatP)0; }
	__device__ floatP mu_01(floatP t) { return GM(t); }
	__device__ floatP bj_00(floatP t) { return t == pensiontime ? bpension : (floatP)0; }
	__device__ floatP bj_01(floatP t) { return (floatP)0; }
public:
	#define PESTATES 1
	__device__ void dV(floatP t, floatP *V, floatP *res){ 
		res[0] = r(t) * V[0] - b_0(t) - mu_01(t) * (0 - V[0] + bj_01(t));
	}

	__device__ void bj_ii(floatP t, floatP *res){
		res[0] = bj_00(t);
	}
} pe;
\end{lstlisting}

Note that this code uses the $floatP$ type alias to easily switch between float and double precision.
What may be of interest is the that the $pe$ object is copied to device memory and that a compile-time replaced $PESTATES$ macro is defined.
The device-size object is then referenced in the plan-specific implementation of the Runge-Kutta 4 solver.
The macro is due to C/C++ requiring constant values for array initialization.
Dynamic memory allocation and deallocation of the temporary arrays could be used, but testing showed that it had a significant negative implication on the runtime, especially at higher thread and block configurations.

The results of running one iteration with 32-bit float precisions took 42,0 ms, about 65\% slower than the C\# equivalent.
The double-precision took 124,48, or about 500\% times slower than the C\# equivalent.
The key factor to consider is the fact that the parallelized version was never intended to be faster with one iteration.
Where the single-threaded version scales linearly with the amount of iterations, the parallelized version scales much better.

The highest performance tested with float precision was at 91,43 iterations per millisecond, making it more than 2200 times faster than the C\# version.
The highest performance with double precision was at 25,97 iterations per millisecond, making it more than 650 times faster than the C\# version.
The double precision version, while respectable, was a good bit slower than could be expected and could most likely be optimized to a rate where it would be about half as fast as the float precision equivalent.

For the full float results see table \ref{table:cudacfloattime} and for double results see table \ref{table:cudacdoubletime}.
What may be of special interest is the diagonals where the total thread amount ($blocks \cdot threads$) is the same.
Here a trend can be spotted where increasing the number of threads appears to be preferable to increasing the number of blocks, though not by a large factor.

For the full CUDA C performance comparisons, see appendix \ref{app:cuda_runtimes}.
For comparison of results with the original solution, see section \ref{subsec:result_comparison}.


\begin{table}[h!]
\centering
{\setlength{\extrarowheight}{2pt}{\setlength{\tabcolsep}{3pt}
\begin{tabular}{ | c | c | c | c | c | c | c | c | }
  \hline
\diaghead{Threads/Blocks}{Threads}{Blocks}
           & 1     &   14  & 14*5  & 14*10 & 14*20 & 14*25 & 14*30 \\ \hline
  1        & 0,02  & 0,33  & 1,55  & 1,56  & 2,00  & 1,92  & 2,17  \\ \hline
  8        & 0,19  & 2,65  & 12,39 & 12,45 & 15,94 & 15,35 & 17,35 \\ \hline
  16       & 0,38  & 5,28  & 24,69 & 24,81 & 31,58 & 30,59 & 34,61 \\ \hline
  32       & 0,75  & 10,48 & 49,02 & 49,26 & 63,16 & 60,76 & 68,91 \\ \hline
  64       & 1,47  & 20,58 & 64,17 & 70,13 & 81,71 & 80,77 & 86,85 \\ \hline
  128      & 2,62  & 35,83 & 72,14 & 80,42 & 83,14 & 85,78 & 85,69 \\ \hline
  256      & 3,20  & 44,15 & 78,57 & 90,82 & 91,07 & 88,55 & 89,01 \\ \hline
  512      & 12,16 & 45,49 & 91,03 & 91,35 & 91,41 & 91,40 & 91,43 \\ \hline
\end{tabular}}}
\caption{CUDA C calculations per ms with float precision\label{table:cudacfloattime}}
\end{table}


\begin{table}[h!]
\centering
{\setlength{\extrarowheight}{2pt}{\setlength{\tabcolsep}{3pt}
\begin{tabular}{ | c | c | c | c | c | c | c | c | }
  \hline
\diaghead{Threads/Blocks}{Threads}{Blocks}
           &  1   &   14  & 14*5  & 14*10 & 14*20 & 14*25 & 14*30 \\ \hline
  1        & 0,01 & 0,11  & 0,51  & 0,50  & 0,65  & 0,61  & 0,69  \\ \hline
  8        & 0,06 & 0,90  & 4,06  & 4,01  & 5,16  & 4,89  & 5,50  \\ \hline
  16       & 0,13 & 1,79  & 8,11  & 8,02  & 10,33 & 9,77  & 11,01  \\ \hline
  32       & 0,26 & 3,59  & 16,22 & 16,03 & 20,65 & 19,53 & 22,01 \\ \hline
  64       & 0,51 & 6,94  & 18,49 & 21,71 & 23,59 & 24,10 & 25,12 \\ \hline
  128      & 1,00 & 11,52 & 22,74 & 25,00 & 25,30 & 25,25 & 25,69 \\ \hline
  256      & 1,67 & 12,98 & 25,55 & 25,59 & 25,93 & 25,88 & 25,97 \\ \hline
  512      & 1,85 & 25,95 & 25,96 & 25,97 & 25,97 & 25,97 & 25,97 \\ \hline
\end{tabular}}}
\caption{CUDA C calculations per ms with double precision\label{table:cudacdoubletime}}
\end{table}