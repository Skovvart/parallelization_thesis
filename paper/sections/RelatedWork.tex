% !TeX root = ../thesis.tex
\section{Related Work}\label{sec:relatedwork}
Much of this work is inspired by other projects also done in collaboration with Peter Sestoft of the IT University of Copenhagen and the Actulus project\cite{actulus}.

This includes earlier work done by Nicolai Dahl Blicher-Pedersen and Christian Harrington.
In their bachelor thesis on \emph{The CUDA Life Insurance Reserve Estimator} (CLIRE)\cite{dahlharrington} they worked on the same six example life insurance plans, though not the GF810 collective spouse pension, and among other things implemented them in CUDA C.
Rather than working in F\# with Alea.cuBase, they instead resort to extending their CLIRE Generator to work within a Scala implementation of the Actulus Modeling Language (AML) that produces optimized CUDA C code.
Where they experience a loss in performance when switching to generated code, this project only experienced an increase.
They do however claim to reach much larger performance increases compared to the original single-threaded C\# solution.
Rather than generating kernels for each plan they also use a ``super-kernel'' containing all plans. 
They do conclude in their future improvements section that it was most likely a mistake, which was one factor when it was decided not to attempt creating any ``super kernels'' in this thesis.

Christian Gerhs Kuhre\cite{christiangk} did his bachelor thesis around the same time as this masters thesis.
His thesis contains work on both F\# Alea.cuBase as well as CalcSpec transformation and was very similar in approach if not quite in scope.
It also contained work using parameter arrays much like was done in this thesis, but in his case this was at least partially implemented using F\# unions.
The performance tests performed in this thesis are however a lot more wide-ranging and extensive and the CalcSpec transformation is more mature.
His thesis did also not cover any work on the GF810 collective spouse pension.

S\o{}ren S\o{}nderby Nielsen and Thorkild Burup did work on ``Using GPGPU for computing pension scenarios''\cite{nielsenburup}. 
Among other things they experimented with adaptive, rather than static, step-sizes for the Runge-Kutta solver (Runge-Kutta-Fehlberg).
They also worked on interest rate generation rather than having this statically defined by the insurance plans, and much like Blicher-Pedersen \& Harrington they also worked with a ``super kernel'' containing all life insurance products.
They also did interesting work using structures to define variability in F\#, though the actual GPU code was written in CUDA C.

Niklas Schalck Johansson, Nikolaj Aaes and Hildur Uffe Flemberg worked on ``GPU Optimization of Base Form 820 Collective Artificial Spouse Pension Plan''\cite{aaesjohanssonflemberg} (``grundform 820'' or GF820 in Danish) which is very similar to the GF810 plan (see section \ref{sub:gf810}) but for the fact that the annuity is potentially deferred. 
Rather than working with the suggested model by Klaus Grue, which may not be applicable for the GF820 plan, they instead attempted parallelizing aspects of the original model of computation using CUDA C and achieved a performance increase of up to factor 680.
