% !TeX root = ../thesis.tex
\section{Actulus CalcSpec parsing and code generation}\label{sec:calcspecgeneration}
The previous implementations have been useful enough for comparisons, but can be considered tedious to implement manually and more error prone.
Being able to generate life insurance plan kernels from higher level specifications such as Actulus CalcSpec (see section \ref{subsec:background:calcspec} or \url{https://wiki.actulus.dk/Documentation.CalculationPlatform-CalculationSpecification.ashx} (requires login) for more information) would be much more convenient, efficient and less prone to errors while also potentially benefiting from future optimizations at no cost.

Similar work transforming CalcSpec to Alea.cuBase kernels was also done in Christian Gehrs Kuhre bachelor thesis.
See section \ref{sec:relatedwork} for more information.

The Actulus project provided lexer and parser definitions for Actulus CalcSpec for use with FsLex and FsYacc\cite{fslexfsyacc} which meant no work was spent on transforming the textual representation of CalcSpec to an abstract syntax tree (AST).
To generate a kernel given an abstract syntax tree, all that is required is to convert the Actulus AST to an F\# quotation AST.
This is however not completely straightforward.
CalcSpec focuses on specifying the various coefficients used in the differential equation where as the $dV$ method needs to explicitly insert the coefficients into Thiele's equation for all states. 
The $bj\_ii$ also needs to assign any lump sum payments, which CalcSpec specifies by using the Dirac delta function (see section \ref{subsec:delta}) in the $b\_j$ coefficient of the equations.

To perform the transformation, a series of steps are performed.
At first all of the \emph{expressions} of the CalcSpec are converted to a quotation expression of nested $Let$ definitions.
This is fairly straightforward as it deals strictly with method and lambda declarations (essentially the same thing), constant variables and algebraic expressions.


The $dV$ and $bj\_ii$ lambda-signatures are then added within the final Let-declaration of the expressions.
For the $dV$ method the signature added is \lstinline$(t:floatP)-Z(V:deviceptr<floatP>)->(result:deviceptr<floatP>)$ and for $bj\_ii$ the signature is \lstinline$(t:floatP)->(result:deviceptr<floatP>)$.

The Markov-model (see section \ref{subsec:background:lifeinsurance}) is then generated using maps extracted from the \emph{equations} describing the constant functions (interest rate ($r\_j$) and benefit paid ($b\_j$)) as well as transition probabilities and transition costs ($mu\_jk$ and $b\_jk$).
If any calls to a delta-function (see section \ref{subsec:delta}) of the form \lstinline$factor * delta(expression)$ is spotted in $b\_j$ (the only place it is allowed), it is removed and instead added to the transition cost map from and to the state with the form \lstinline$if expression = 0 then factor else zero$.
This is done as the Dirac delta-function is in this case much more efficiently expressed in the $bj\_ii$ method.
Thiele's differential equation (see section \ref{subsec:background:thielerungekutta}) is then assigned to all states using the value specified in the various maps in the $dV$ method, and bj\_jk values where j equals k is assigned to all states in the $bj\_ii$ method.
Finally, the entire quotation AST is reduced using an expression reduction method that performs some arithmetic reductions.
See section \ref{subsec:exprReduction} for more information on this method.

The steps the CalcSpec goes through can be seen in code samples \ref{pe_cs} to \ref{pe_quote_format}.
\clearpage
\begin{lstlisting}[language=calcspec, caption=Pure endowment in CalcSpec, label=pe_cs]
calculation = {
  name = 'Pure endowment',
  algorithm = { type = 'Runge Kutta 4', parameters={ stepsize = 0.01 }},
  equations = {
    0 = { r_j = r, b_j = b0, mu_jk = { 1 = GM }, b_jk = { } },
    1 = { },
  },
  range = { from = 40, to = 0 },
  boundaryvalues = { 0 = 0 , 1 = 0 },
  expressions = {
    interestrate = 0.05,
    bpension = 1,
    pensiontime = 35,
    age = 30,
    r(t) = interestrate,
    b0(t) = bpension * delta(t - pensiontime),
    GM(t) = 0.0005 + 10 ˆ (5.728 - 10 + 0.038*(age + t))
  }
}
\end{lstlisting}

\begin{lstlisting}[language=fsharp, caption=Abstract Syntax Tree for the Pure endowment CalcSpec, label=pe_ast]
( Key.Id("calculation"), Value.Structured(
    (Key.Id("name"), Value.String("Pure endowment"))::
    (Key.Id("algorithm"), Value.Structured(
        (Key.Id("type"), Value.String("Runge Kutta 4"))::
        (Key.Id("parameters"), Value.Structured((Key.Id("stepsize"), Value.Expr(Expr.Number(0.01)))::[])))::[])::
    (Key.Id("equations"), Value.Structured((Key.Number(0), Value.Structured(...:[]))::[]))::
    ...::
    (Key.Id("expressions"), Value.Structured(
      (Key.Id("interestrate"), Value.Expr(Expr.Number(0.05)))::
      ...::
      (Key.Function("r", "t"), Value.Expr(Expr.Id("interestrate")))::
      ...::[]))
    ::[])
)
\end{lstlisting}

\begin{lstlisting}[language=fsharp, caption=Quotation AST for dV method generated from CalcSpec AST, label=pe_quote]
Let("interestrate", Value(0.05), //values converted to type of floatP
  Let("bpension", Value(1.0), 
    Let("pensiontime", Value(35.0), 
      Let("age", Value(30.0), 
        Let("r", Lambda("t", Var("interesterate")), 
          Let("b0", Lambda("t", Call (None, op_Multiply, Var("bpension")::[Application(Var("delta"), Call (None, op_Subtraction, Var("t")::[Var("pensiontime")]))])), 
            Let("GM", Lambda("t", Call (None, op_Addition, Value(0.0005)::[...])), 
              Lambda("t", Lambda("V", Lambda("res", PropertySet (Var("res"), Item, [Value (0),
                      Call (None, op_Subtraction,
                        [Call (None, op_Multiply, [
                          Application(Var("r"), [Var("t")]),
                          PropertyGet (Some (V), Item, [Value (0)])]),
                        Call (None, op_Multiply, [
                          Application(Var("GM"), [Var("t")]),
                          PropertyGet (Some (Var("V")), Item, [Value (0)])])])]))))))))))))
\end{lstlisting}

\begin{lstlisting}[language=fsharp, caption=Formatted quotation AST for dV and bj\_ii methods, label=pe_quote_format]
let dV = 
  <@
    let interestrate = 0.05
    let bpension = 1.0
    let pensiontime = 35.0
    let age = 30.0
    let r = fun t-> interestrate
    let b0 = fun t-> 0.0 //transformed into bj_ii
    let GM = fun t-> 0.0005+exp(2.30258512*(5.728-10.0+0.038*(age+t))) 
    // power expression converted to exponential expression
    fun t V res->
      res.[0] <- r t * V.[0] - GM t * V.[0]
  @>

let bj_ii = 
  <@
    let interesterate = 0.05
    let bpension = 1.0
    let pensiontime = 35.0
    let age = 30.0
    let r = fun t-> interestrate
    let b0 = fun t-> 0.0
    let GM = fun t-> 0.0005+exp(2.30258512*(5.728-10.0+0.038*(age+t)))
    fun t res->
      res.[0] <- if t-pensiontime = zero then bpension else zero
  @>
\end{lstlisting}


The generated kernels used up to 34 registers per thread making 1024 threads per block impossible. 
It may be possible to optimize this at a later point.
The highest number of calculation per ms in the single precision version for all plans was was \emph{98.32}.
See table \ref{table:cubaseGeneratedfloattime} for the full single precision results. 
The highest number of calculations per ms for double precision was \emph{25.56}.
See table \ref{table:cubaseGenerateddoubletime} for the full double precision results. 
For the full generated results see appendix \ref{app:cuBase_calcspec_runtimes}.

\begin{table}[h!]
\centering
{\setlength{\extrarowheight}{2pt}{\setlength{\tabcolsep}{3pt}
\begin{tabular}{ | r | r | r | r | r | r | r | r | }
  \hline
\diaghead{Threads/Blocks}{Threads}{Blocks}
		&	1		&	14		&	14*5	&	14*10	&	14*20	&	14*25	&	14*30	\\ \hline
1		&	0.02	&	0.33	&	1.57	&	1.63	&	2.13	&	2.02	&	2.34	\\ \hline
8		&	0.19	&	2.65	&	12.56	&	12.99	&	16.96	&	16.13	&	18.67	\\ \hline
16		&	0.38	&	5.28	&	25.01	&	25.81	&	33.97	&	32.08	&	37.15	\\ \hline
32		&	0.75	&	10.50	&	49.54	&	51.09	&	66.98	&	63.73	&	73.85	\\ \hline
64		&	1.49	&	20.68	&	66.79	&	70.38	&	89.60	&	85.59	&	95.05	\\ \hline
128		&	2.96	&	36.42	&	77.60	&	78.88	&	85.01	&	89.53	&	92.46	\\ \hline
256		&	5.35	&	46.03	&	95.02	&	91.57	&	97.84	&	95.32	&	95.63	\\ \hline
512		&	6.66	&	47.92	&	97.20	&	97.89	&	98.18	&	96.17	&	98.26	\\ \hline
1024	&	6.88	&	95.69	&	97.83	&	98.13	&	98.26	&	98.29	&	98.32	\\ \hline
\end{tabular}}}
\caption{CalcSpec generated F\# Alea.cuBase calculations per ms with single precision\label{table:cubaseGeneratedfloattime}}
\end{table}

\begin{table}[h!]
\centering
{\setlength{\extrarowheight}{2pt}{\setlength{\tabcolsep}{3pt}
\begin{tabular}{ | r | r | r | r | r | r | r | r | }
  \hline
\diaghead{Threads/Blocks}{Threads}{Blocks}
	&	1		&	14		&	14*5	&	14*10	&	14*20	&	14*25	&	14*30	\\ \hline
1	&	0.01	&	0.10	&	0.47	&	0.48	&	0.61	&	0.58	&	0.67	\\ \hline
8	&	0.06	&	0.84	&	3.79	&	3.82	&	4.86	&	4.65	&	5.35	\\ \hline
16	&	0.12	&	1.67	&	7.56	&	7.67	&	9.71	&	9.29	&	10.68	\\ \hline
32	&	0.24	&	3.34	&	15.10	&	15.34	&	19.41	&	18.57	&	21.38	\\ \hline
64	&	0.48	&	6.47	&	20.30	&	21.52	&	23.99	&	22.91	&	24.60	\\ \hline
128	&	0.93	&	11.01	&	22.45	&	24.48	&	24.82	&	24.71	&	25.30	\\ \hline
256	&	1.58	&	12.54	&	24.96	&	24.30	&	25.51	&	25.11	&	25.56	\\ \hline
512	&	1.80	&	25.18	&	25.44	&	25.48	&	25.50	&	25.50	&	25.50	\\ \hline
\end{tabular}}}
\caption{CalcSpec generated F\# Alea.cuBase calculations per ms with double precision\label{table:cubaseGenerateddoubletime}}
\end{table}

The single precision results are very similar to the manual implementation.
The double precision results are a negligible amount better than the manual implementation without parameters, but significantly worse than the manual solution with parameters.
This makes sense as the generated $dV$ and $bj\_ii$ methods end up being very similar to the manual implementations without parameters.

\subsection{Parameterization}
To parameterize the kernels generated from CalcSpec, the lambda signatures have to re-add the parameter array as in section \ref{sub:manual_parameterization}.
The \emph{expression} $Let$ bindings are then moved inside the lambda-declaration, and based on a parameter specification, named constants are replaced with positions in the parameter array.

Single precision could perform up to \emph{98.23} calculations per ms while double precision could perform up to \emph{25.91} calculations per ms.
See tables \ref{table:cubaseGeneratedParamfloattime} and \ref{table:cubaseGeneratedParamdoubletime} for the remaining results.
For the full generated parameterized results see appendix \ref{app:cuBase_calcspec_params_runtimes}.

\begin{table}[h!]
\centering
{\setlength{\extrarowheight}{2pt}{\setlength{\tabcolsep}{3pt}
\begin{tabular}{ | r | r | r | r | r | r | r | r | }
  \hline
\diaghead{Threads/Blocks}{Threads}{Blocks}
	&	1		&	14		&	14*5	&	14*10	&	14*20	&	14*25	&	14*30	\\ \hline
1	&	0.02	&	0.33	&	1.55	&	1.64	&	2.11	&	2.01	&	2.33	\\ \hline
8	&	0.19	&	2.63	&	12.41	&	13.08	&	16.79	&	16.07	&	18.78	\\ \hline
16	&	0.37	&	5.24	&	24.72	&	25.78	&	33.36	&	32.01	&	37.52	\\ \hline
32	&	0.74	&	10.39	&	48.76	&	50.69	&	66.18	&	63.31	&	74.14	\\ \hline
64	&	1.48	&	20.43	&	65.31	&	69.70	&	87.92	&	86.67	&	95.34	\\ \hline
128	&	2.93	&	35.94	&	74.53	&	82.03	&	91.19	&	91.34	&	93.82	\\ \hline
256	&	5.28	&	45.44	&	89.98	&	92.04	&	96.83	&	94.90	&	97.53	\\ \hline
512	&	6.61	&	62.37	&	97.11	&	97.82	&	98.12	&	98.15	&	98.23	\\ \hline
\end{tabular}}}
\caption{Parameterized CalcSpec calculations per ms with single precision\label{table:cubaseGeneratedParamfloattime}}
\end{table}

\begin{table}[h!]
\centering
{\setlength{\extrarowheight}{2pt}{\setlength{\tabcolsep}{3pt}
\begin{tabular}{ | r | r | r | r | r | r | r | r | }
  \hline
\diaghead{Threads/Blocks}{Threads}{Blocks}
	&	1		&	14		&	14*5	&	14*10	&	14*20	&	14*25	&	14*30	\\ \hline
1	&	0.01	&	0.10	&	0.48	&	0.48	&	0.62	&	0.59	&	0.68	\\ \hline
8	&	0.06	&	0.84	&	3.82	&	3.85	&	4.99	&	4.69	&	5.45	\\ \hline
16	&	0.12	&	1.68	&	7.61	&	7.74	&	9.96	&	9.37	&	10.90	\\ \hline
32	&	0.24	&	3.35	&	15.19	&	15.48	&	19.92	&	18.73	&	21.78	\\ \hline
64	&	0.48	&	6.50	&	20.29	&	21.80	&	25.19	&	23.65	&	25.78	\\ \hline
128	&	0.93	&	11.11	&	19.50	&	25.60	&	25.93	&	25.62	&	25.91	\\ \hline
256	&	1.60	&	12.78	&	25.26	&	25.93	&	25.98	&	25.97	&	26.00	\\ \hline
512	&	1.83	&	25.54	&	25.84	&	25.88	&	25.90	&	25.91	&	25.91	\\ \hline
\end{tabular}}}
\caption{Parameterized CalcSpec calculations per ms with double precision\label{table:cubaseGeneratedParamdoubletime}}
\end{table}
%\clearpage
\subsubsection{Parameter generation}
A helper method was written to generate the parameter arrays for the parameterized solutions.
As input it takes a parameter specification consisting of a list of tuples containing the constant variable to be parameterized, a start and end value as well as the step size of the parameters.
For every parameter in the specification, a list of numbers going from the start to the end value in intervals of the step size is generated.
To generate the final parameter array the cartesian product of all these lists is then computed.
A map from variable name to relative position in the parameter array based on the list-position in the parameter specification is also returned.
For an example of the generation, see code sample \ref{gen_params}.

\begin{lstlisting}[language=fsharp, caption=Parameter generation with the age variable ranging from 40 to 42 and the interest rate variable ranging from 8\% to 10\%, label=gen_params]
mkParams(("age",40.0, 42.0, 1.0)::[("interestrate",0.08, 0.10, 0.01)])
//returns [40.0; 0.08; 40.0; 0.09; 40.0; 0.1; 41.0; 0.08; 41.0; 0.09; 41.0; 0.1; 42.0; 0.08; 42.0; 0.09; 42.0; 0.1]
//total of 18 values with 2 parameters = 9 iterations
\end{lstlisting}

Finally, a runtime configuration (number of blocks and number of threads per block) is suggested based on the total number of iterations.
A runtime configuration should ideally have the number of threads be a multiple of the warp-size (32) and ideally at least the amount of SMs as the number of blocks.
The generated parameters might require less than 448 iterations (14 SMs times 32 threads).
The suggestion is based off a few factors that attempt to make sure that at least a warp and the number of SMs is used.
First it is checked if the number of iterations is less than a warp in which case one block and the number of iterations as the number of threads per block will be suggested.
If higher than a warp but less than a warp times the number of SMs, the rounded up number of the iterations divided by 32 as blocks and 32 threads per block.
In case the number of iterations is higher than this, it will try continually increasing the number of threads per block by 32 until it reaches 512 or the iterations are less than or equal to the found number of threads times the number of SMs.
If this is still not enough, the number of blocks will be increased until it can satisfy iterations enough for all the parameters.
Further work could very well improve this by viewing it as as an optimization problem.

This will still potentially leave us in the situation than the total number of iterations is less than the number of threads per block times blocks.
This is somewhat unavoidable, so the RK4\_n kernel was altered to skip thread-IDs greater than the total number of parameterized iterations.
