% !TeX root = ../thesis.tex
\subsection{Result discussion}\label{subsec:result_comparison}
Results of running the six plans described in section \ref{subsubsec:background:insuranceplans} using the Runge-Kutta four solver were gathered for both single and double floating point precision using the initial C\# solution.
It was then the goal for further solutions to match these results with a low margin of error.

The six plans in total output 398 numbers that are compared. 
Many of them are always zero, making the total number of actual comparisons 255.
It is worth noting that all results are printed with a decimal component of 16 digits, though .NET single precision floats when printed will only show up to 7 decimal digits with trailing zeros for the remaining digits.
The CUDA C solution however prints digits for all 16 decimal digits even for single precision floats.
This means that the lowest error tolerance possible for comparisons where both results are doubles is 1e-16 (to the 16th digit) and 1e-7 if comparing single-precision results.
It is also important to note that number of errors is often inflated as one error in a plan will typically propagate to the rest of the output of the plan.

A tool was created to report the number of errors between two result-outputs with a certain error tolerance.
See table \ref{table:errorComparison} for the results of running this tool.

All the solutions are identical down to the 4th decimal digit, all doubles down to the 13th and all singles down to the 5th.
It also seems that the two .NET solutions (C\# and F\# Alea.cuBase) are a lot more similar than the CUDA C solution.
This indicates that a very potential source of error could be different representations of floating point numbers in the .NET and CUDA C languages that are then passed to the kernels executed on the GPU.

The .NET solutions are practically identical for double precision, and for single precision the rate of error at the 6th decimal digit is also a lot lower than the CUDA C solution.

The double to single comparison shows that the relative error of the results seem very similar between the different floating point number precisions.

%mål standard afvigelse?

\begin{table}[h!]
\centering
{\setlength{\extrarowheight}{2pt}{\setlength{\tabcolsep}{3pt}
\begin{tabular}{ | l | r | r | r | r | r | r | r | r | } \hline
\diaghead{Errors/Tolerance}{Errors}{Tolerance}
							&	1e-16	&	1e-15	&	1e-14	&	1e-13	&	1e-7	&	1e-6	&	1e-5	&	1e-4	\\ \hline
C\# 64 versus CUDA C 64		&	190		&	97		&	36		&	0		&	0		&	0		&	0		&	0		\\ \hline
C\# 64 versus cuBase 64		&	2		&	0		&	0		&	0		&	0		&	0		&	0		&	0		\\ \hline
cuBase 64 versus CUDA C 64	&	190		&	97		&	36		&	0		&	0		&	0		&	0		&	0		\\ \hline
%							&			&			&			&			&			&			&			&			\\ \hline
C\# 64 versus C\# 32		&			&			&			&			&	205		&	120		&	24		&	0		\\ \hline
cuBase 64 versus cuBase 32	&			&			&			&			&	202		&	122		&	24		&	0		\\ \hline
CUDA C 64 versus CUDA C 32	&			&			&			&			&	198		&	120		&	25		&	0		\\ \hline
%							&			&			&			&			&			&			&			&			\\ \hline
C\# 32 versus CUDA C 32		&			&			&			&			&	118		&	34		&	0		&	0		\\ \hline
C\# 32 versus cuBase 32		&			&			&			&			&	34		&	14		&	0		&	0		\\ \hline
cuBase 32 versus CUDA C 32	&			&			&			&			&	103		&	34		&	0		&	0		\\ \hline
\end{tabular}}}
\caption{Number of unequal results with various error tolerances.\label{table:errorComparison}}
\caption*{32 = single precision, 64 = double precision}
\end{table}

\subsubsection{Timing}
All of the previous timing results are based on a 100-iteration average of the kernel execution time.
They are however only measuring the time the kernel took to execute.
This does not include initialization and copying of memory from and to the device, nor the time it takes to do runtime compilation of the kernels.
The memory aspect seems to be fairly negligible, as the highest recorded time was 779.5 ms for a parameterized kernel execution with 1024 threads and 350 blocks.
For the full memory copying timing results, see appendix \ref{app:cuBase_manual_params_memcpy}.

The compilation of kernels also takes some time, though it is a one-time operation that leads to a reusable kernel.
Compilation of the parameterized kernels takes about 1560 ms.
As compilation is a one-time operation, this result is not based on a 100 iteration average, but should be fairly reflective of the time required at least for insurance plans with a low number of states.

The last aspect that is not timed is the generation of parameters.
As this is something that can be expected to be provided rather than generated, and would take the same time to generate even if not provided, this has not been timed.

\subsubsection{Parameterization}
Parameterization was tested by manually altering constants in the original C\# solution and comparing the output to the output of a parameterized kernel. 
Ignoring similar error rates as reported previously, the results were identical indicating that parameterization works as intended.